% PLan for reflective analysis, paragraph for each, maybe 2 for last point, aim for ~1pg
% What went well? (Unit testing, integration testing of independent functions)
% What did not go well? (Integration of dependent functions)
% Compare to bachelors (did what I learn carry forward?, how do the projects differ?)
% What could I have done better (Verification that accuracy remains similar +-5%, more rigourous testing, learning more about tensorflow 1 and 2 near the start of the project, followed test driven development rigour better)
\section{Reflection}
This project has been an enjoyable process for me.
It has allowed me to expand my knowledge of different machine learning models and techniques.
I usually use an iterative Waterfall approach to software development, using unit tests has been a welcome change of pace.
It made the overall development of the project a lot smoother as it helped me understand the purpose of each function used in the learning process, and how they interact between each other.

The overall process of the development of the parallel version was smooth. 
This was due to the unit tests making it easier to see where problems were and how they may affect future model development over different iterations.
By using unit tests in this project, I have found that there are more benefits to using unit tests than not using them, which I will be using in future projects.

When considering what did not go well, it was mainly the integration of each function into the overall system.
This is because of the difference between Tensorflow and NumPy with different implementations of functions, such as the dot product.
This caused a variety of solutions to be created to manage this, however they all had some variation that means they may not have been optimal.
Another problem from the project is that the speed up seen in experimental results is not as expected, with approximately 2x speed up occurring at 1000+ neurons, and approximately 7x speed up at 3000+ neurons, whereas we were expecting to see around a 100x speed up with larger network sizes.

In comparison to my bachelors project, I feel this project has been more successful. This was likely due to the extra time that could be dedicated to the project, without other assignments to worry about.
Another reason for this project being more successful is that I was able to take lessons learned from my bachelors project, such as involving my supervisor more and using more resources available to me.

If I was to redo this project, I would focus more on understanding different parallel frameworks before beginning development, and how different versions of these libraries affect their performance.
Another thing that I would redo is to be more rigorous in following test driven development, as I have developed a parallel function before the tests.
This had a negligible effect on the development of the software, but does encourage poor habits when developing code.