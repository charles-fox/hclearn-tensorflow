
\subsection{Unit tests}
\label{subsec:unittest}
{
A big problem with creating and improving software is that it is likely to break.
This is especially true when refactoring to speed up the code.
This is due to changes in underlying implementation creating side effects in the code.
These side effects cause the code to break, and if left unchecked increases the technical debt, which accrues as a natural part of the software development life-cycle.

\begin{minipage}{\linewidth}
\begin{lstlisting}[caption=Example unit to be tested, label=code:unit, language=python]
def boltzmannProbs(W, x, axis=0):
    """
    :param W: Input Weight Matrix
    :param x: Input Vector Matrix
    :param axis: Axis of contraction
    :return: activation probabilities
    """
    reward = tf.tensordot(W, x, axis)
    E_on  = tf.negative(reward)
    E_off = 0.0*E_on
    Q_on = tf.math.exp(-E_on)
    Q_off = tf.math.exp(-E_off)
    P_on = tf.math.divide(Q_on, tf.math.add(Q_on, Q_off))
    return P_on
\end{lstlisting}
\end{minipage}

Unit tests are a method of both reducing technical debt and ensuring that behaviour remains the same.
This is because the tests are used to verify behaviour after refactoring a unit.
A unit can be considered any functional object, however we define a unit as a custom function in the code.
An example of a unit can be seen in \coderef{code:unit}.
}



{
A test is a wrapper function that provides known inputs to the unit and checks the output of the unit against a known output.
The check is done via an assert statement.
An assert statement fails if the values that the assertion is checking do not match.
We define a test function as a singular concept on the unit being tested.
This allows us to understand the nature of the unit for that concept.
It also reduces confusion on what a given test is meant to do.
This reduces technical debt as future developers are not having to figure out what the 
purpose of a given test is, which increases productivity.
An example of a test can be seen in \coderef{code:test}
}



\begin{minipage}{\linewidth}
\begin{lstlisting}[caption=Example test for a unit, label=code:test, language=python]
def test_rbm_small():
    input_weights = np.array([[.1, .2], [.3, .4]])
    input_vector = np.array([.5, .6])
    calculated_results = rbm.boltzmannProbs(input_weights, input_vector)
    actual_results = np.array([0.54239794, 0.5962827])
    assert_almost_equal(calculated_results, actual_results,
                            err_msg="[BoltzmannProbs] Calculated results are not equal to actual results to 7 decimal places.")
\end{lstlisting}
\end{minipage}

{
%A big problem with the improvement of software is that it breaks when you change it. You are trying to speed it up but u change its functionality.   [introduce a problem]
%[sets you up to describe a solution]
%Units tests are an amazing way to fix this problem.  
%what is a unit test? 
%    what is a unit?
%        a function?  a class?  a file ?
%    what is a test?
%        one call of a unit? many calls of a unit?
%            many calls with a similar theme?
%        Used for testing how functions handle inputs and whether actual behaviour is what we expect.
%                using assertions
%                    say what is an assertion
%        show code examples  
%    martin fowler defined unit test as ...
 
 }      


\subsubsection{Test Driven Development and HC-UCPF}
Test Driven Development is a method of developing software that focuses on writing tests before writing code.
This allows the developer to write valid code the first time round, and does not need to spend long hours debugging the code.
This works well for new projects.
However our project involves refactoring an existing project.
Thus we can not follow test driven development to the letter.
But we can use the approach of test driven development as part of the refactoring process.
This means we write our tests and ensure that the existing code passes them.
This also has the benefit of helping to understand the purpose of the code.

By refactoring code in this manner, it helps with the process of debugging.
This is due to being able to pass the system inputs to the tests, and seeing what is causing the error.
Furthermore, from our definition of the unit test, we can obtain a hierarchical approach to debugging.
This is due to each of our functions being units, and invoking other functional units within our system.

For example, the function \verb|learn| in learnWeights.py\footnote{Found at \url{https://github.com/A-Yakkus/hclearn/blob/msc2020/learnWeights.py\#L16}} invokes the \verb|boltzmannProbs| function in rbm.py\footnote{Found at \url{https://github.com/A-Yakkus/hclearn/blob/msc2020/rbm.py\#L9}}.
If an error occurs during the learn function with respect to the \mono{boltzmannProbs} function, we can pause execution of the learn function and run the current inputs through our tests to find where the erroneous behaviour occurs.
Once we understand the erroneous behaviour, it can be considered trivial to fix it.
This would then allow successive invocations of the system to run without encountering the error. 

    % (though disputed by other people x y and z in these ways..)
    % what they do:
    %     they assure us that each unit works
    %     that allows us to try changing a unit
    %             eg to optimise
    %     it also allows delta debugging:
    %         hierarchy of unit tests
    %         find a bug by looking at red/green and homeing in
    %     what this is great
    %         saving tons of time! instant bug detection and localistion
    % TDD: write tests first
    % Refactoring: fowler's approach: how to optimise other peoples code  (vs TDDm is different)
    %     % Our case has it different as we already have functions implemented.
    % Parallel debugging:
    %   parallel is especially important use case here
    %         [NB we already told the reader about parallel techs, so we can refer to that now]
    %   new big idea here -- using fowler refactors to make parallel debugging tolerable !  is  big deal.

